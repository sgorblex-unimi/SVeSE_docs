%% Copyright (C) 2022 Alessandro "Sgorblex" Clerici Lorenzini and Edoardo "Miniman" Della Rossa
%
% This work may be distributed and/or modified under the
% conditions of the LaTeX Project Public License, either version 1.3
% of this license or (at your option) any later version.
% The latest version of this license is in
%   http://www.latex-project.org/lppl.txt
% and version 1.3 or later is part of all distributions of LaTeX
% version 2005/12/01 or later.
%
% This work has the LPPL maintenance status `maintained'.
%
% The Current Maintainer of this work is Alessandro Clerici Lorenzini
%
% This work consists of the files listed in work.txt


\chapter{Introduzione}

Il Sistema di Voto e Scrutinio Elettronico, del quale in seguito si presentano i requisiti, ha come obiettivo quello di permettere il corretto svolgimento di elezioni, anche di tipo diverso tra loro, sia con votazione remota, sia con votazione presso un seggio fisico.

Analogamente alle elezioni "tradizionali", e in conformità con le \href{https://www.interno.gov.it/sites/default/files/2021-07/linee_guida_voto_elettronico_decreto_9.7.2021.pdf}{direttive ministeriali}, il Sistema è progettato per garantire le proprietà fondamentali del voto (come segretezza e anonimato) e per rispettare le normative vigenti riguardanti il trattamento di informazioni sensibili. 

Oltre a queste proprietà, il Sistema fornisce anche la possibilità di inserire delle utenze speciali, chiamate Garanti, pensate per verificare e controllare che l'amministrazione elettorale sia corretta e trasparente. 

Dal punto di vista della fruibilità, è fornita per il voto on-line un'interfaccia chiara e comprensibile, adatta all'utilizzo anche da parte dei meno esperti. Non sono previste modifiche al processo di voto nei seggi fisici dal punto di vista del Votante, con l'unica eccezione della mancanza del supporto fisico delle schede, che sono digitali al fine di permettere una raccolta unica dei voti e uno scrutinio immediato.

Anche grazie a ciò, l'Amministratore della Sessione di Voto è in grado di consultare immediatamente i risultati a Sessione conclusa. Anche i Garanti hanno accesso ai risultati, per verificare che non vengano comunicati in modo alterato dall'Amministratore.

Il Sistema garantisce la possibilità di scegliere diverse modalità di voto (voto ordinale, categorico, categorico con preferenze e referendum) e diverse modalità di selezione del vincitore (maggioranza, maggioranza assoluta, referendum senza quorum, referendum con quorum).

Dal punto di vista della cybersecurity, il Sistema è progettato in modo da garantire che tutti i vincoli sulla sicurezza di voti e delle informazioni sensibili vengano garantiti tramite tecnologie apposite.

\section{Glossario}
Viene fornito un glossario con i termini più utilizzati in seguito, per permettere al lettore una corretta comprensione del contesto.

\subsection{Persone e utenze}
\begin{description}
	\item[Elettore] avente diritto al voto;
	\item[Votante] elettore che esercita il diritto di voto.
\end{description}

\subsubsection{Utenze di gestione}
\begin{description}
	\item[Amministratore della Sessione di Voto] utenza unica, costruisce le schede e aggiunge i garanti;
	\item[Garanti (della Sessione di Voto)] enti che verificano la correttezza della Sessione di Voto e hanno accesso ai risultati;
	\item[Responsabile di Seggio (Fisico)] responsabile che attivi e monitori il Sistema in un Seggio Fisico.
\end{description}


\subsection{Termini specifici}
\begin{description}
	\item[Scheda (Elettorale)] supporto digitale che contiene le possibili scelte di voto e registra le scelte del votante. Sostituisce la scheda elettorale cartacea fino al momento dell'invio del suffragio;
	\item[Seggio Fisico] luogo fisico dove l'elettore che decida di votare di persona esercita il suo diritto; unico e predeterminato per elettore;
	\item[Sessione di Voto] insieme di procedure che contribuiscono a finalizzare una consultazione tramite il Sistema, dall'apertura alla chiusura del voto da parte dell'Amministratore;
	\item[Sistema (di Voto e Scrutinio Elettronico)] insieme degli applicativi software e hardware che gestiscono il procedimento di voto;
	\item[Suffragio] scelta del votante conforme alle modalità dell'elezione.
\end{description}




