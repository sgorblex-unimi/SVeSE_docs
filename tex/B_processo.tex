%% Copyright (C) 2022 Alessandro "Sgorblex" Clerici Lorenzini and Edoardo "Miniman" Della Rossa
%
% This work may be distributed and/or modified under the
% conditions of the LaTeX Project Public License, either version 1.3
% of this license or (at your option) any later version.
% The latest version of this license is in
%   http://www.latex-project.org/lppl.txt
% and version 1.3 or later is part of all distributions of LaTeX
% version 2005/12/01 or later.
%
% This work has the LPPL maintenance status `maintained'.
%
% The Current Maintainer of this work is Alessandro Clerici Lorenzini
%
% This work consists of the files listed in work.txt


\chapter{Note di processo}\label{processo}




\section{Divisione dei compiti}
In questa sezione sono raccolti i compiti svolti da ciascun membro, divisi per ambito.


\subsection{Collettivi}
\begin{description}
	\item[Relazione] progettazione e stesura dei requisiti.
\end{description}


\subsection{Alessandro Clerici Lorenzini}
\begin{description}
	\item[Relazione] descrizione dell'implementazione (capitolo \ref{implementazione} eccetto \ref{gui}, \ref{implclassi} e \ref{deployment}), appendici \ref{utilizzo} e \ref{processo};
	\item[Organizzazione] gestione della struttura del progetto e del build tool (e.g. \verb!pom.xml!), integrazione, git flow e gestione repository;
	\item[Implementazione] model e model-side controller, sicurezza e login, persistenza, testing, logging, utility script, vincoli.
\end{description}


\subsection{Edoardo Della Rossa}
\begin{description}
	\item[Relazione] diagrammi di progettazione (capitolo \ref{diagrammi}), descrizione dell'interfaccia grafica (paragrafo \ref{gui}), diagrammi delle classi di programma (\ref{implclassi}) e di deployment (\ref{deployment});
	\item[Implementazione] view e view-side controller.
\end{description}




\section{Sviluppi futuri}
In questa sezione si raccolgono spunti per possibili sviluppi futuri del progetto SVeSE.


\subsection{Modello}
\begin{description}
	\item[Gestione dei Seggi Fisici] si vedano i requisiti funzionali al paragrafo \ref{reqfunz}, in aggiunta alla reperibilità dei risultati basati su area geografica;
	\item[Schede bianche] una modellazione ad-hoc delle schede bianche per le elezioni che le supportano; al momento il supporto è possibile creando un'apposita \verb!Choice!;
	\item[Supporto alle immagini] a ogni \verb!Choice! può essere associata un'immagine;
	\item[Stato sessione] meglio distinguere lo stato di sessione non \emph{ready} e sessione conclusa.
\end{description}


\subsection{Interfaccia web}
\begin{description}
	\item[Schede elettorali multiple] il supporto grafico per sessioni di più di una scheda elettorale, già supportate dal modello;
	\item[Miglior supporto alle eccezioni] controllare l'input prima di comunicare con il modello oppure fornire output specifico per l'eccezione sollevata;
	\item[Pagina principale] fornire una pagina di benvenuto invece di proiettare l'utente direttamente nella schermata di voto;
	\item[Timer di login] forzare il logout di un utente dopo un dato intervallo di tempo;
	\item[Accesso all'auditing] rendere visibili i log da parte di Amministratore e Garanti;
	\item[Miglioramenti vari] dialoghi di conferma ed errore più chiari, migliore interfaccia di logout, \dots.
\end{description}


\subsection{Persistenza}
\begin{description}
	\item[Persistenza] rendere persistente la sessione nella sua interezza;
	\item[Cache] implementare un sistema di caching dei dati comunemente richiesti e difficili da computare come i risultati; potenzialmente persistente.
\end{description}




\section{Tecnologie utilizzate}


\subsection{Framework e librerie}
\begin{itemize}
	\item Hibernate
	\item JUnit
	\item SLF4J
	\item Spring Boot
	\item Spring Data JPA
	\item Spring Security
	\item Spring Web
	\item Vaadin
\end{itemize}


\subsection{Software}
\begin{description}
	\item[Interpreti/linguaggi] Java, sh
	\item[Build tool] Maven
	\item[Database] PostgreSQL
	\item[UML] PlantUML (\url{https://plantuml.com})
	\item[Version control] Git + GitKiss (\url{https://github.com/sgorblex/GitKiss})
	\item[Stesura relazione] \LaTeX
	\item[Editor] Neovim con Java LSP (Language Server Protocol)
	\item[Browser] Firefox (testato anche su Chromium)
	\item[Sistemi operativi] Arch Linux
\end{description}
