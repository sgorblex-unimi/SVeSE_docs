\documentclass{report}
\usepackage[italian]{babel}
\usepackage{geometry}
\usepackage{graphicx}
\usepackage{caption}

\geometry{a4paper,top=2cm,bottom=3cm,left=3cm,right=3cm,heightrounded,bindingoffset=5mm}

\begin{document}

\title{Requirements Analysis and Specification Document \\
\large Sistema di Voto e Scrutinio Elettronico}
\author{Alessandro Clerici Lorenzini, Edoardo Della Rossa}
\date{Anno Accademico 2021-2022}
\maketitle

\tableofcontents

\chapter{Introduzione}
\section{Obiettivi}
\section{Destinatari}
\section{Scopo del sistema}
\section{Glossario (definizioni, acronimi e abbreviazioni)}
\begin{description}
	\item[Amministratore della Sessione di Voto] costruisce le schede e aggiunge i garanti;
	\item[Elettore] avente diritto al voto;
	\item[Garanti] enti che verificano la correttezza della Sessione di Voto e hanno accesso ai risultati;
	\item[Responsabile di Seggio] responsabile che attivi e monitori il Sistema in un seggio fisico;
	\item[Seggio (fisico)] luogo fisico dove l'elettore che decida di votare di persona esercita il suo diritto;
	\item[Sessione di Voto] insieme di procedure che contribuiscono a finalizzare una consultazione tramite il Sistema.;
	\item[Sistema] (S maiuscola) Sistema di Voto e Scrutinio Elettronico;
	\item[Votante] elettore che esercita il diritto di voto.
\end{description}


\chapter{Introduzione}
\section{Descrizione generale}
\section{Caratteristiche del prodotto}
\section{Ambiente operativo}
\section{Documentazione per l'utente}
\section{Assunzioni, vincoli e dipendenze}

\chapter{Specifica dei requisiti}
\section{Requisiti funzionali}
\section{Requisiti di interacce esterne}
\subsection{Interfacce utente}
\subsection{Interfaccie hardware}
\subsection{Interfacce di comunicazione}
\section{Requisiti non funzionali}
\subsection{Requisiti di prestazione}
\subsection{Attributi di qualità del software}
\subsection{Requisiti di sicurezza}

\chapter{Appendice}

\end{document}
