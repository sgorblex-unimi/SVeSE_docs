\documentclass{report}
\usepackage[italian]{babel}
\usepackage[utf8]{inputenc}
\usepackage{caption}
\usepackage{geometry}
\usepackage{graphicx}
\usepackage[hidelinks]{hyperref}

\geometry{a4paper,top=2cm,bottom=3cm,left=3cm,right=3cm,heightrounded,bindingoffset=5mm}
\setcounter{secnumdepth}{3}



\begin{document}

\title{Requirements Analysis and Specification Document \\
\large Sistema di Voto e Scrutinio Elettronico}
\author{Alessandro Clerici Lorenzini, Edoardo Della Rossa}
\date{Anno Accademico 2021-2022}

\maketitle
\tableofcontents






\chapter{Introduzione}




\section{Obiettivi}




\section{Destinatari}




\section{Scopo del sistema}




\section{Glossario}


\subsection{Persone e utenze}
\begin{description}
	\item[Elettore] avente diritto al voto;
	\item[Votante] elettore che esercita il diritto di voto.
\end{description}

\subsubsection{Utenze di gestione}
\begin{description}
	\item[Amministratore della Sessione di Voto] utenza unica, costruisce le schede e aggiunge i garanti;
	\item[Garanti (della Sessione di Voto)] enti che verificano la correttezza della Sessione di Voto e hanno accesso ai risultati;
	\item[Responsabile di Seggio (Fisico)] responsabile che attivi e monitori il Sistema in un Seggio Fisico.
\end{description}


\subsection{Termini specifici}
\begin{description}
	\item[Scheda (Elettorale)] supporto digitale che contiene le possibili scelte di voto e registra le scelte del votante. Sostituisce la scheda elettorale cartacea fino al momento dell'invio del suffragio;
	\item[Seggio Fisico] luogo fisico dove l'elettore che decida di votare di persona esercita il suo diritto; unico e predeterminato per elettore;
	\item[Sessione di Voto] insieme di procedure che contribuiscono a finalizzare una consultazione tramite il Sistema, dall'apertura alla chiusura del voto da parte dell'Amministratore;
	\item[Sistema (di Voto e Scrutinio Elettronico)] insieme degli applicativi software e hardware che gestiscono il procedimento di voto;
	\item[Suffragio] scelta del votante conforme alle modalità dell'elezione.
\end{description}



\chapter{Descrizione generale}
Il sistema di voto e scrutinio elettronico del quale in seguito di presentano i requisiti ha come obiettivo quello di permettere il corretto svolgimento di elezioni, anche di tipo diverso tra loro, sia con votazione remota, sia con votazione presso un seggio fisico.
Analogamente alle elezioni "tradizionali", e seguendo le \textit{direttive ministeriali},
% (TODO aggiungere link)
il sistema è progettato per garantire le proprietà fondamentali del voto (come segretezza e anonimato) e per rispettare tutte le relative normative riguardanti il trattamento di informazioni sensibili. 
Oltre a queste proprietà, il sistema fornisce anche la possibilità di inserire delle utenze speciali, chiamate garanti, pensate per verificare e controllare che l'amministrazione elettorale sia corretta e trasparente. 
\par

Dal punto di vista della fruibilità, oltre che un interfaccia chiara e comprensibile, adatta all'utilizzo anche da parte dei meno esperti, non sono previste modifiche al processo di voto nei seggi fisici.
L'unico cambiamento effettuato, è il suppororto fisico delle schede, che diventa digitale per permettere una raccolta unica dei voti e uno scrutinio immediato.
In questo modo, l'amministratore della sessione di voto è in grado di consultare immediatamente i risultati. Anche i garanti hanno accesso ai risultati, per verificare che non vengano comunicati in modo alterato dall'amministratore della sessione di voto.
\par
Il Sistema garantisce diverse modalità di voto (voto ordinale, categorico, categorico con preferenze e referendum) e diverse modalità con cui viene selezionato il vincitore (maggioranza, maggioranza assoluta, referendum senza quorum, referendum con quorum).
Dal punto di vista della cybersecurity, il Sistema è progettato in modo da garantire che tutti i vincoli sulla sicurezza di voti e informazioni sensibili venga garantito tramite tecnologie apposite.


\section{Caratteristiche del prodotto}




\section{Ambiente operativo}




\section{Documentazione per l'utente}




\section{Assunzioni, vincoli e dipendenze}






\chapter{Specifica dei requisiti}




\section{Requisiti funzionali}

\subsection{Amministratore della Sessione di Voto}
L'Amministratore della Sessione di Voto si occupa di aggiungere i Garanti (predeterminati con meccanismi esterni al Sistema) e di costruire le schede elettorali previste per la consultazione.

Una volta ricevuta la verifica da parte dei Garanti, l'Amministratore può dare il via alla Sessione di Voto, durante la quale gli elettori possono esprimere i rispettivi suffragi. L'amministratore inserisce nel sistema i dettagli sulla Sessione di Voto, compresa la sua durata. Quest'ultima è regolata da un timer allo scadere del quale la Sessione è conclusa.

L'amministratore ha la facoltà di sospendere la Sessione qualora una situazione eccezionale lo richieda. Eventuali richieste di sospensione sono inviate all'Amministratore dai Responsabili di Seggio. Durante una sospensione l'Amministratore ha facoltà di terminare definitivamente la Sessione, tuttavia tale procedimento è da intendersi eccezionale.

\subsection{Garanti}
I Garanti si occupano di assicurare che l'amministrazione elettorale sia corretta e trasparente.

Prima dell'inizio della sessione hanno il compito di controllare che l'Amministratore abbia impostato correttamente i parametri della Sessione di Voto e che le nomine dei garanti siano coerenti con quanto stabilito dai regolamenti.

Alla fine della sessione hanno il diritto di consultare i risultati per verificarne la proclamazione, e hanno libero accesso ai dati di monitoraggio per garantire che non ci siano stati malfunzionamenti.


\subsection{Modalità di voto}
Le seguenti modalità di voto devono essere disponibili durante la creazione della scheda:
\begin{description}
	\item[Voto ordinale] all'elettore è richiesto di ordinare le scelte (candidati o gruppi) presenti nella scheda in base alle proprie preferenze;
	\item[Voto categorico] l'elettore inserisce una preferenza per non più di una scelta (candidato o gruppo);
	\item[Voto categorico con preferenze] l'elettore inserisce una preferenza per un gruppo e ha la possibilità di indicare una o più preferenze tra le scelte del gruppo selezionato;
	\item[Referendum] consiste in una domanda fatta all'elettorato con la quale si chiede se si sia favorevoli o contrari a un determinato quesito.
\end{description}

In tutte le modalità deve essere garantita la possibilità di astenersi.


\subsection{Modalità di selezione del vincitore}
La selezione del vincitore può avvenire secondo diverse modalità, impostate dall'Amministratore conformemente all'elezione.
\begin{description}
	\item[maggioranza] il vincitore è il candidato che ha ottenuto il maggior numero di voti;
	\item[maggioranza assoluta] il vincitore è il candidato che ha ottenuto la maggioranza assoluta dei voti, cioè il 50\% + 1 dei voti espressi;
	\item[referendum senza quorum] si procede al conteggio dei voti indipendentemente se abbia partecipato o meno alla consultazione la maggioranza degli aventi diritto al voto; 
	\item[referendum con quorum] si procede al conteggio dei voti espressi solo nel caso in cui abbia partecipato alla consultazione la maggioranza degli aventi diritto al voto.
\end{description}


\subsection{Certificazione della Sessione di Voto}
Un organismo indipendente opportunamente nominato deve verificare la validità e il funzionamento del sistema prima che esso entri in funzione


\subsection{User stories}

\subsubsection{Elettore}
Come elettore devo poter...
\begin{itemize}
	\item poter votare on-line;
	\item poter votare di persona (sapendo in che seggio).
\end{itemize}

\subsubsection{Votante}
Come votante devo poter...
\begin{itemize}
	\item poter contare sull'anonimia e segretezza del mio voto;
	\item poter selezionare in modo facile le mie preferenze secondo la modalità di voto della/e scheda/e.
\end{itemize}

\subsubsection{Amministratore della Sessione di Voto}
Come Amministratore della Sessione di Voto devo poter...
\begin{itemize}
	\item aggiungere schede con le varie modalità e scelte per ogni scheda;
	\item avviare e fermare la Sessione di Voto (+ sospensioni);
	\item consultare i risultati delle elezioni.
\end{itemize}

\subsubsection{Garante}
Come Garante della Sessione di Voto devo poter...
\begin{itemize}
	\item verificare le schede e la lista dei garanti prima dell'avvio della Sessione di Voto;
	\item consultare i risultati delle elezioni.
\end{itemize}

\subsubsection{Responsabile di Seggio}
Come Responsabile di Seggio devo poter...
\begin{itemize}
	\item dopo l'identificazione, la postazione per i singoli votanti;
	\item verificare l'avvenuto voto di ciascun votante (senza ovviamente vederne il contenuto).
\end{itemize}




\section{Requisiti di interfacce esterne}


\subsection{Interfacce utente}


\subsection{Interfacce hardware}


\subsection{Interfacce di comunicazione}




\section{Requisiti non funzionali}


\subsection{Voto}

\subsubsection{Diritto di voto}
Ogni elettore deve poter esercitare il suo diritto al voto, ossia diventare votante. La possibilità di votare deve essere garantita dal Sistema sia di persona in un Seggio Fisico, sia da remoto via Internet.

\subsubsection{Anonimato del voto}
In nessun modo deve essere possibile collegare un suffragio con l'identità che l'ha espresso.

\subsubsection{Segretezza del voto}
In nessun modo deve essere possibile collegare un elettore con il suffragio che ha espresso.

\subsubsection{Imparzialità del sistema}
Il Sistema non deve influenzare in alcun modo la decisione del votante.

\subsubsection{Unicità del voto}
Ogni elettore può votare una e una sola volta per consultazione.

\subsubsection{Regole sull'espressione del voto}
\begin{enumerate}
	\item Il votante può accedere solo alle schede per le elezioni in cui gode dell'elettorato attivo;
	\item Il votante può esprimere solo un voto valido (eventualmente esprimendo più votazioni se le modalità lo richiedono) oppure può astenersi. Non è consentito esprimere un voto nullo;
	\item Il votante deve poter verificare tramite l'interfaccia grafica che il voto espresso sia coerente con la propria volontà (per evitare errori inconsapevoli), dopodiché poterlo confermare e ricevere un messaggio di avvenuto suffragio;
	\item Il Sistema non deve fornire alcuna prova del contenuto del suffragio dopo la conferma dello stesso, ai fini di evitare ogni possibile utilizzo illecito anche per conto terzi.
\end{enumerate}

\subsubsection{Raccolta dei voti}
Dopo l'espressione del singolo suffragio, il voto deve pervenire all'urna del sistema centrale senza alterazioni. Il conteggio avviene solo dopo la chiusura della Sessione di Voto.


\subsection{Attinenza alle specifiche ufficiali dello Stato}
Il Sistema deve essere conforme alle \textit{Linee  guida  per  la  sperimentazione  di  modalità  di  espressione  del  voto  in  via digitale} pubblicate dal Ministero dell'Interno il 25 maggio 2021


\subsection{Requisiti di prestazione}


\subsection{Attributi di qualità del software}


\subsection{Requisiti di sicurezza}

\subsubsection{Autenticazione delle utenze di gestione}
Ogni utente di gestione, ovvero Amministratore, Garante o Responsabile di Seggio, deve autenticarsi presso il Sistema prima di accedervi, al fine di permettere l'implementazione di meccanismi efficaci di access control. 

Non ci sono vincoli sullo specifico metodo di autenticazione utilizzato: potranno essere utilizzati dei sistemi di identità digitale pubblici, così come un sistema utente/password opportunamente configurato.

\subsubsection{Autenticazione degli elettori}
Ai fini di garantire l'autenticità dei dati raccolti, è fondamentale che ogni elettore si identifichi. Il Sistema deve prevedere un sistema di autenticazione per consentire solamente agli elettori la possibilità di votare (e solo una volta).

On-line l'autenticazione avviene tramite un sistema analogo a quello per l'autenticazione delle utenze di gestione. Nei seggi fisici l'autenticazione avviene tramite la presentazione di un documento d'identità.

\subsubsection{Availability e malfunzionamenti}
Il sistema deve essere sviluppato con un'attenzione particolare nei confronti dell'availability. Se il Sistema ha un interruzione momentanea nel funzionamento, deve informare immediatamente il votante, e bloccare le procedure di voto, riprendendole successivamente nel caso dopo aver ripetuto la procedura di identificazione.

\subsubsection{Privacy}
Ogni informazione sensibile che che viene memorizzata deve essere trattata secondo le direttive e le norme vigenti. Sul dispositivo usato non deve comunque rimanere alcuna traccia del contenuto del suffragio espresso.

\subsubsection{Auditing}
Ai fini di mantenere il maggior livello di sicurezza possibile, opportuni sistemi di auditing (log) devono essere forniti, i quali non devono comunque contenere il contenuto dei suffragi.






\chapter{Appendice}

\end{document}
